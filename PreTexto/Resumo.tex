\newpage

\mychapter{0}{Resumo}\label{resumo}
Atzingen, G. V. v. \textbf{Simulação, controle e automação de um forno tipo túnel utilizando tecnologia embarcada.} \pageref{LastPage} f. Qualificação (Doutorado) - Faculdade de Zootecnia e Engenharia de Alimentos, 2014.

\noindent
\\Baseado na grande evolução dos dispositivos eletrônicos nos últimos 35 anos e dos novos hardwares de baixo custo e alto poder computacional, esta tese tem como objetivo apresentar um software e hardware de controle e simulação em tempo real utilizando um sistema embarcado de hardware livre de baixo custo que garanta a qualidade na produção de alimentos em fornos tipo túnel. Para isto, modelagem matemática e simulação do perfil de temperatura dentro do forno foram realizadas para que o sistema de controle possa ter informação da temperatura no alimento em tempo real, contando apenas com os sensores fixos no forno. A informação desta simulação alimenta o controle PID, garantindo que o perfil de temperatura desejado para o aquecimento/cozimento do alimento seja obedecido, melhorando a qualidade do produto final.

\par
\vspace{1em}
\noindent\textbf{Palavras-chave:} Diferenças Finitas, CFD, Raspberry Pi, Arduino