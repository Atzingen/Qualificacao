% Limpa cabeçalhos.
% (solução para lidar com a númeração das páginas pré-textuais).
\pagestyle{empty}

%% Capa
\begin{titlepage}

% Se quiser uma figura de fundo na capa ative o pacote wallpaper
% e descomente a linha abaixo.
% \ThisCenterWallPaper{0.8}{nomedafigura}

\begin{center}
{\LARGE \nomedoaluno}
\par
\vspace{200pt}
{\Huge \titulo}
\par
\vfill
\textbf{{\large Pirassununga, SP}\\
{\large \the\year}}
\end{center}
\end{titlepage}

% Faz com que a página seguinte sempre seja ímpar (insere pg em branco)
%\cleardoublepage

% Numeração em elementos pré-textuais é opcional (ativada por padrão).
% Para desativá-la comente a linha abaixo.
%\pagestyle{fancy}

% Números das páginas em algarismos romanos
%\pagenumbering{roman}

%% Página de Rosto

% Numeração não deve aparecer na página de rosto.
\thispagestyle{empty}

\begin{center}
{\LARGE \nomedoaluno}
\par
\vspace{200pt}
{\Huge \titulo}
\end{center}
\par
\vspace{90pt}
\hspace*{175pt}\parbox{7.6cm}{{\large Qualificação apresentada à Faculdade de Zootecnia e Engenharia de Alimentos da Universidade de São Paulo, como parte dos requisitos para a obtenção de Título de Doutor em Ciências, na Área de Engenharia de Alimentos.}}

\par
\vspace{1em}
\hspace*{175pt}\parbox{7.6cm}{{\large Orientador: Ernane José Xavier Costa}}

\par
\vfill
\begin{center}
\textbf{{\large Pirassununga, SP}\\
{\large \the\year}}
\end{center}

\newpage

% Ficha Catalográfica
%\hspace{8em}\fbox{\begin{minipage}{10cm}
%Atzingen, Gustavo Voltani von.

%\hspace{2em}\titulo

%\hspace{2em}\pageref{LastPage} páginas

%\hspace{2em}Qualificação apresentada a Faculdade de Zootecnia e Engenharia de Alimentos da Universidade de São Paulo, como %parte dos requisitos para a obtenção de Título de Doutor em Ciências, na Área de Engenharia de Alimentos.

%Area de Concentração: Ciências da Engenharia de Alimentos

%Orientador: Ernane José Xavier Costa

%\begin{enumerate}
%\item Diferenças Finitas
%\item CFD
%\item Raspberry Pi
%\item Arduino
%\end{enumerate}


%\end{minipage}}
%\vspace{2em}

%\newpage
%\pagestyle{fancyplain}
% Desabilitar protrusão para listas e índice
%\microtypesetup{protrusion=false}
