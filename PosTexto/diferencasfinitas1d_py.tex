\lstset{language=Python}
\begin{lstlisting}
from __future__ import division
import numpy as np
import matplotlib.pyplot as plt
import pylab
import time

###################### Funcoes #########################################
def tridiagonal(A,B):
    '''
    Programa que resolve a equacao AX = B, recebendo como entrada
    as matrizes A e B e retornando X.
    Precondicoes: 
    A matriz A deve ser quadrada (n x n)
    A matriz B deve ser uma matriz coluna (n x 1)
    A matriz A deve ser tridiagonal
    '''
    [Alinha,Acoluna] = A.shape
    [Blinha,Bcoluna] = B.shape
    if ( Acoluna !=  Alinha):  # Checa se a matriz A e quadrada
        print ' A Matriz A nao e quadrada'
        return
    if ( Alinha !=  Blinha):# Checa se AxB existe
        print ' A x B nao e uma operacao matricial permitida'
        return
    if ( Bcoluna != 1 ): # Checa se B e uma matriz coluna
        print ' A matriz B nao possui apenas uma coluna'
        return    
    # Checando se a matriz e tridiagonal:
    for i in range(Alinha):           # Checa a parte superior da matriz
        for j in range(i+2,Alinha):
            if ( A[i,j] != 0 ):
                print 'A matriz nao e tridiagonal ' ,  'A[', i , ',' , j , ']'
                return
        for j in range(0,i-2):
            if ( A[i,j] != 0 ):
                print 'A matriz nao e tridiagonal ' , 'A[', i , ',' , j , ']'
                return       
    ## Metodo TDMA
    print ' Todas as condicoes corretas ! '
    x = np.zeros([Alinha,1])
    print A, B
    for i in range(1,Alinha):
        a1 = A[i,i-1]
        a2 = A[i-1,i-1]
        m = a1/a2
        A[i,i] = A[i,i]   - (m*A[i-1,i])
        B[i]   = B[i]     - (m*B[i-1])
    x[Alinha-1] = B[Alinha-1] / A[Alinha-1,Alinha-1]
    
    for k in range(Alinha-2,-1,-1):
        x[k] = ( B[k] - (A[k,k+1]*x[k+1]) )/A[k,k]
    return x

######################## Condicoes do problema #######################
L0 = 0
Lf = 1
n  = 100
deltax  = (Lf - L0)/n
deltax2 = (Lf - L0)/(n+1)  # verificar n+1
x       = np.linspace(L0,Lf,n)
f       = 100*np.exp(x)
T0      = 20
Tf      = 60

####################### Metodo Direto ################################
tic = time.time()
A = np.zeros([n,n])
B = np.zeros([n,1])
B1 = np.zeros([n,1])
B2 = np.zeros([n,1])
B2[0]   = T0
B2[n-1] = Tf

for i in range(n):
    B1[i] = f[i]

B = -(deltax2)*(deltax2)*B1 - B2

for i in range(n):
    for j in range(n):
        if ( j == i - 1 ):
            A[i,j] = 1
        elif ( j == i ):
            A[i,j] = -2
        elif ( j == i + 1 ):
            A[i,j] = 1
        else:
            A[i,j] = 0
print B.size
Y = tridiagonal(A,B)
toc = time.time() - tic
plt.figure(1)
plt.plot(x,Y)
#pylab.show()

###################### Metodos Iterativos ############################

iter = 2000
x = np.linspace(L0,Lf,n)
J = T0 + (Tf-T0)*x
J[0]   = T0
J[n-1] = Tf
J2 = T0 + (Tf-T0)*x
J2[0]   = T0
J2[n-1] = Tf
J3 = T0 + (Tf-T0)*x
J3[0]   = T0
J3[n-1] = Tf

###################### Gauss - Seidel ################################

tic2 = time.time()
for k in range (iter):
    for i in range(1,n-2):
        J[i] = (J[i-1] + J[i+1] + deltax*deltax*f[i])/2
toc2 = time.time() - tic2

###################### Jacobi  ########################################

tic3 = time.time()
for k in range (iter):
    for i in range(1,n-2):
        Jant  = J2
        J2[i] = (J2[i-1] + J2[i+1] + deltax*deltax*f[i])/2
toc3 = time.time() - tic3

###################### SOR  ########################################

tic4 = time.time()
for k in range (iter):
    for i in range(1,n-2):
        Jn = (J3[i-1] + J3[i+1] + deltax*deltax*f[i] )/2   
        J3[i] = J3[i] + 1.7*( Jn - J3[i] )
toc4 = time.time() - tic4

#######################  Jacobi - numpy ############################
J4 = T0 + (Tf-T0)*x
J4[0]   = T0
J4[n-1] = Tf
tic5 = time.time()
for k in range (iter):
    J4[1:n-2] = ( J4[0:n-3] + J4[2:n-1] + deltax*deltax*f[1:n-2] )/2
toc5 = time.time() - tic5

print 'Metodo Direto: ' , toc
print 'Gauss Seidel:  ' , toc2
print 'Jacobi:        ' , toc3
print 'SOR:           ' , toc4
print 'Jacobi f2py:   ' , toc5
plt.plot(x,J)
plt.plot(x,J2)
plt.plot(x,J3)
pylab.show()
\end{lstlisting}