\chapter{Apêndice}\label{apêndice}

\section{Diferenças Finitas 1D - Vários métodos}\label{alg:dif1d}

Programa que implementa a solução da equação de difusão de calor em 1 Dimensão utilizando os métodos Direto, Jacobi, Gauss-Seidel, e SOR, utilizando uma condição de contorno que permite encontrar a solução analítica para fins de conparação. Programa implementado em Python2.7

\singlespacing
\lstset{language=Python}
\begin{lstlisting}
from __future__ import division
import numpy as np
import matplotlib.pyplot as plt
import pylab
import time

###################### Funcoes #########################################
def tridiagonal(A,B):
    '''
    Programa que resolve a equacao AX = B, recebendo como entrada
    as matrizes A e B e retornando X.
    Precondicoes: 
    A matriz A deve ser quadrada (n x n)
    A matriz B deve ser uma matriz coluna (n x 1)
    A matriz A deve ser tridiagonal
    '''
    [Alinha,Acoluna] = A.shape
    [Blinha,Bcoluna] = B.shape
    if ( Acoluna !=  Alinha):  # Checa se a matriz A e quadrada
        print ' A Matriz A nao e quadrada'
        return
    if ( Alinha !=  Blinha):# Checa se AxB existe
        print ' A x B nao e uma operacao matricial permitida'
        return
    if ( Bcoluna != 1 ): # Checa se B e uma matriz coluna
        print ' A matriz B nao possui apenas uma coluna'
        return    
    # Checando se a matriz e tridiagonal:
    for i in range(Alinha):           # Checa a parte superior da matriz
        for j in range(i+2,Alinha):
            if ( A[i,j] != 0 ):
                print 'A matriz nao e tridiagonal ' ,  'A[', i , ',' , j , ']'
                return
        for j in range(0,i-2):
            if ( A[i,j] != 0 ):
                print 'A matriz nao e tridiagonal ' , 'A[', i , ',' , j , ']'
                return       
    ## Metodo TDMA
    print ' Todas as condicoes corretas ! '
    x = np.zeros([Alinha,1])
    print A, B
    for i in range(1,Alinha):
        a1 = A[i,i-1]
        a2 = A[i-1,i-1]
        m = a1/a2
        A[i,i] = A[i,i]   - (m*A[i-1,i])
        B[i]   = B[i]     - (m*B[i-1])
    x[Alinha-1] = B[Alinha-1] / A[Alinha-1,Alinha-1]
    
    for k in range(Alinha-2,-1,-1):
        x[k] = ( B[k] - (A[k,k+1]*x[k+1]) )/A[k,k]
    return x

######################## Condicoes do problema #######################
L0 = 0
Lf = 1
n  = 100
deltax  = (Lf - L0)/n
deltax2 = (Lf - L0)/(n+1)  # verificar n+1
x       = np.linspace(L0,Lf,n)
f       = 100*np.exp(x)
T0      = 20
Tf      = 60

####################### Metodo Direto ################################
tic = time.time()
A = np.zeros([n,n])
B = np.zeros([n,1])
B1 = np.zeros([n,1])
B2 = np.zeros([n,1])
B2[0]   = T0
B2[n-1] = Tf

for i in range(n):
    B1[i] = f[i]

B = -(deltax2)*(deltax2)*B1 - B2

for i in range(n):
    for j in range(n):
        if ( j == i - 1 ):
            A[i,j] = 1
        elif ( j == i ):
            A[i,j] = -2
        elif ( j == i + 1 ):
            A[i,j] = 1
        else:
            A[i,j] = 0
print B.size
Y = tridiagonal(A,B)
toc = time.time() - tic
plt.figure(1)
plt.plot(x,Y)
#pylab.show()

###################### Metodos Iterativos ############################

iter = 2000
x = np.linspace(L0,Lf,n)
J = T0 + (Tf-T0)*x
J[0]   = T0
J[n-1] = Tf
J2 = T0 + (Tf-T0)*x
J2[0]   = T0
J2[n-1] = Tf
J3 = T0 + (Tf-T0)*x
J3[0]   = T0
J3[n-1] = Tf

###################### Gauss - Seidel ################################

tic2 = time.time()
for k in range (iter):
    for i in range(1,n-2):
        J[i] = (J[i-1] + J[i+1] + deltax*deltax*f[i])/2
toc2 = time.time() - tic2

###################### Jacobi  ########################################

tic3 = time.time()
for k in range (iter):
    for i in range(1,n-2):
        Jant  = J2
        J2[i] = (J2[i-1] + J2[i+1] + deltax*deltax*f[i])/2
toc3 = time.time() - tic3

###################### SOR  ########################################

tic4 = time.time()
for k in range (iter):
    for i in range(1,n-2):
        Jn = (J3[i-1] + J3[i+1] + deltax*deltax*f[i] )/2   
        J3[i] = J3[i] + 1.7*( Jn - J3[i] )
toc4 = time.time() - tic4

#######################  Jacobi - numpy ############################
J4 = T0 + (Tf-T0)*x
J4[0]   = T0
J4[n-1] = Tf
tic5 = time.time()
for k in range (iter):
    J4[1:n-2] = ( J4[0:n-3] + J4[2:n-1] + deltax*deltax*f[1:n-2] )/2
toc5 = time.time() - tic5

print 'Metodo Direto: ' , toc
print 'Gauss Seidel:  ' , toc2
print 'Jacobi:        ' , toc3
print 'SOR:           ' , toc4
print 'Jacobi f2py:   ' , toc5
plt.plot(x,J)
plt.plot(x,J2)
plt.plot(x,J3)
pylab.show()
\end{lstlisting}
\doublespacing
%%%%%%%%%%%%%%%%%%%%%%%%%%%%%%%%%%%%%%%%%%%%%%%%%%%%%%%%%%%%%%%%%%%%%%%%%%%%%%%%%%%%%%%%%%%%%%%%%%%%%%%%%%%%%%%%%%%

\section{Diferenças Finitas - Várias Linguagens}\label{alg:linguagens}

Compara o tempo de execução do mesmo Algoritmo de Diferenças Finitas em várias linguagens (Python, Matlab, Octave, Fortran90 e c). O algoritmo utilizado como teste é o método SOR com 300 iterações.

Fortran:

\singlespacing
\lstset{language=Fortran}
\begin{lstlisting}
! Temperatura 1d - Funcao conhecida - metodo SOR 
! Calculo do tempo de execucao
implicit none
real start, finish, L0, Lf, deltax, T0, Tf, sor, i2, k2, a, Nz2
integer i, Nz, k
parameter (Nz = 100)
real J(0:Nz-1)
real Jn(0:Nz-1)
real u(0:Nz-1)
real x(0:Nz-1)
real f(0:Nz-1)
integer p
do p =1, 20
	call cpu_time(start)
	L0 = 0
	Lf = 1
	T0 = 20
	Tf = 60
	deltax = (Lf - L0)/Nz 
	sor = 1.9

	do i = 0, Nz-1 ! Preenchendo os valores de x ( de 0 a 1 )
		i2 = real(i)
		Nz2= real(Nz)
		x(i) = i2/Nz2
	enddo

	do i = 0, Nz-1 ! Preenchendo os valores do aquecimento f(x)
		a = x(i)
		f(i) = exp(a)*100
	enddo


	do i = 0, Nz-1 ! Preenchendo os valores iniciais de J 
		J(i) = T0 + ( Tf - T0 )*x(i)
	enddo

	do k = 0, 3000 ! Calculo de J (SOR)
		do i = 1, Nz-2
			Jn(i) = ( J(i-1) + J(i+1) + deltax*deltax*f(i))/2
			J(i) = J(i) + sor*( Jn(i) - J(i) )	
		enddo
	enddo

	!open (unit=2,file='arquivo.txt',status='unknown')	
	!do i = 0, Nz-1
	!	write (2,20) J(i)
	!	write (*,*) J(i)
	!enddo
	!20	format(0P,F7.1,2(3X,1P,E10.4))

	call cpu_time(finish)
	print '("Time = ",ES14.7," seconds.")',finish-start
enddo
end 
\end{lstlisting}

C:

\lstset{language=c}
\begin{lstlisting}
/*
Temperatura 1d - Funcao conhecida - metodo SOR
*/

#include <stdio.h>
#include <math.h> 
#include <time.h>

int main(int argc, char** argv)
{
	clock_t begin, end; // Definindo as variaveis doproblema
	double time_spent;
	float start, finish, L0, Lf, deltax, T0, Tf, sor, i2, k2, a, Nz2;
	int i, k, iter, p;
	int Nz = 100;
	float J[Nz], Jn[Nz], u[Nz], x[Nz], f[Nz];
	for (p = 1; p < 21; p++)
	{
		begin = clock();
		L0 = 0;
		Lf = 1;
		T0 = 20;
		Tf = 60;
		deltax = (Lf - L0)/Nz; 
		sor = 1.9;

		for (i=0; i<Nz; i++) // Preenchendo os valores de x ( de 0 a 1 )
		{
			i2   = (float)i;
			Nz2  = (float)Nz;
			x[i] = i2/Nz2;
		}
		for (i=0; i<Nz; i++) // Preenchendo os valores do aquecimento f(x)
		{
			a    = x[i];
			f[i] = exp(a) * 100.0;
		}
		for (i=0; i<Nz; i++) // Preenchendo os valores iniciais de J 
		{
			J[i] = T0 + ( Tf - T0 )*x[i];
		}

		for (iter=0; iter < 3001; iter++)
		{
			for (i=1; i< Nz-1; i++)
			{
				Jn[i] = ( J[i-1] + J[i+1] + deltax*deltax*f[i])/2.0;
				J[i] = J[i] + sor*( Jn[i] - J[i] );
			}
		}
		end = clock();
		time_spent = (double)(end - begin) / CLOCKS_PER_SEC;
		printf(" tempo: %f \n",time_spent);
	}
	/*for (i=0; i<Nz; i++) // Imprimindo o resultado
	{
		printf("T = %f\n",J[i]);
	}*/
}
\end{lstlisting}

Matlab:

\lstset{language=Matlab}
\begin{lstlisting}
%Temperatura 1d - Funcao conhecida - metodo SOR  
%Calculo do tempo de execucao
L0 = 0;   % posicao inicial da barra
Lf = 1;  % posicao final (tamanho) da barra - em metros
n = 100;  % dividindo o problema em 1000 elementos
deltax = (Lf-L0)/n;
%deltax para o metodo direto
deltax2=(Lf-L0)/(n+1);
x = linspace(L0,Lf,n)';
f = 100*exp(x);
T0 = 20;  % temperatura em x=0 (inicio da barra) em kelvin
Tf = 60;  % temperatura em x=Lf (final da barra) em kelvin
J3 = T0 + (Tf-T0)*x;
J3(1) = T0;
J3(n) = Tf;
for p=1:20
	id = tic;
	for k=1:3000
	    for i=2:n-1     
	       Jn   = (J3(i-1) + J3(i+1) + deltax*deltax*f(i))/2;
	       J3(i) = J3(i) + 1.9*(Jn - J3(i));
	    end
	end
	toc(id)
end
\end{lstlisting}

Matlab Vetorizado:

\lstset{language=Matlab}
\begin{lstlisting}
    % parte principal do algoritmo vetorizada
    for k=1:3000
	    for i=2:n-1     
	       Jn   = (J3(i-1) + J3(i+1) + deltax*deltax*f(i))/2;
	       J3(i) = J3(i) + 1.9*(Jn - J3(i));
	    end
    end
	 
\end{lstlisting}

Python:

\lstset{language=Python}
\begin{lstlisting}

# -*- coding: latin-1 -*-
'''  Temperatura 1d - Funcao conhecida - metodo SOR  
Calculo do tempo de execucao
'''  
from __future__ import division
import numpy as np
import matplotlib.pyplot as plt
import pylab
import time

for p in range(1,21):
	tic = time.time()
	####### Condicoes do problema #######################
	L0 = 0
	Lf = 1
	n  = 100
	deltax  = (Lf - L0)/n
	deltax2 = (Lf - L0)/(n+1)  # verificar n+1
	x       = np.linspace(L0,Lf,n)
	f       = 100*np.exp(x)
	T0      = 20
	Tf      = 60
	u = (100*np.exp(1)-60)*x+120-100*np.exp(x)# Solucao analitica
	k = 1.9                    # Overelaxacao
	########################################
	J = T0 + (Tf-T0)*x      # Temperatura inicial linear
	J2 = J
	erro = 1000
	for l in range(1,3000):
		for i in range(1,n-1):
			Jn = ( J[i-1] + J[i+1] + deltax*deltax*f[i])/2
			J[i] = J[i] + k*( Jn - J[i] )	
	toc = time.time()
	print toc - tic
#plt.plot(J)
#pylab.show()

\end{lstlisting}

Python vetorizado com Numpy

\lstset{language=Python}
\begin{lstlisting}
    # alteracao do loop principal com vetorizacao
	for l in range(1,3000):
		J2[1:n-2] = (J[0:n-3] + J[2:n-1] + (deltax*deltax*f[1:n-2]))/2
		J[1:n-2] = J[1:n-2] + k*(J2[1:n-2] - J[1:n-2])
\end{lstlisting}

\doublespacing
%%%%%%%%%%%%%%%%%%%%%%%%%%%%%%%%%%%%%%%%%%%%%%%%%%%%%%%%%%%%%%%%%%%%%%%%%%%%%%%%%%%%%%%%%%%%%%%%%%%%%%%%%%%%%%%%%%%%%

\section{Parâmetro $w$ do método SOR}\label{alg:sor}

O código abaixo apresenta uma varredura de valores de w entre 1 e 2, passando por 200 posições. Para cada valor de w o algoritmo repete as iterações até que se encontre um erro mínimo e então guarda quantas iterações foram necessárias para cada valor de w.

\singlespacing
\lstset{language=Python}
\begin{lstlisting}
# -*- coding: latin-1 -*-
"""
Criado em: 03/030/2014
1 D - SOR - identificacao do melhor fator de OR
1d heat transfer stead state 
CC. T(0) = 20 , T(1) = 60 , L = 1m , f(x) = 100e^x  
Resolucao analítica:  u(x) = (100e - 60)x + 120 - 100e^x 
"""
from __future__ import division
import numpy as np
import matplotlib.pyplot as plt
import pylab
import time

######################## Condicoes do problema #######################
L0 = 0
Lf = 1
n  = 100
deltax  = (Lf - L0)/n
deltax2 = (Lf - L0)/(n+1)  # verificar n+1
x       = np.linspace(L0,Lf,n)
f       = 100*np.exp(x)
T0      = 20
Tf      = 60
######################## Solucao analítica ##########################
u = (100*np.exp(1) - 60)*x + 120 - 100*np.exp(x)
total = np.sum(u)

erro = 1000
J = T0 + (Tf-T0)*x
J[0]   = T0
J[n-1] = Tf

xi = np.linspace(1,1.99,200)
y  = np.zeros([50,1])
j = 0
for k in np.linspace(1,1.99,200):
	contador = 0
	erro = 1000
	J = T0 + (Tf-T0)*x
	J[0]   = T0
	J[n-1] = Tf
	while erro > 100:
		contador = contador + 1
		for i in range(1,n-2):
			Jn = (J[i-1] + J[i+1] + deltax*deltax*f[i] )/2
			J[i] = J[i] + k*( Jn - J[i] )
			erro = np.abs(np.sum((J-u)*(u-J)))
		if contador > 1000:
			break
	y[j] = contador
	j = j + 1
	print contador, k    

plt.plot(xi,y)
pylab.show()
\end{lstlisting}
\doublespacing
%%%%%%%%%%%%%%%%%%%%%%%%%%%%%%%%%%%%%%%%%%%%%%%%%%%%%%%%%%%%%%%%%%%%%%%%%%%%%%%%%%%%%%%%%%%%%%%%%%%%%%%%%%%%%%%%%%%%%

\section{Controle PID}

O código abaixo apresenta a classe PID, que será instanciada no controle. Foi testado com interpretador Python2.7 em Linux Ubuntu 13.10, Windows 7 e Mac OS X e não necessita de nenhuma biblioteca adicional.

\singlespacing
\lstset{language=Python}
\begin{lstlisting}
# -*- coding: latin-1 -*-
'''
Classe de controle PID Discreto
Baseado no programa desenvolvido por http://code.activestate.com/recipes/577231-discrete-pid-controller/
sobre a licena MIT: http://opensource.org/licenses/MIT
'''

class PID:
	"""
	Controle PID Discreto
	Este objeto deve ser instanciado com os valores de Kp, Ki e Kd (padrao = 2, 0.5 e 1) e com o set_point
	set_point -> Valor desejado para o controle
	Limite mximo e mínimo para o integrador = 100 e -100 respectivamente
	Integrador e Derivador iniciam com valor 0
	"""
	def __init__(self, P=2.0, I=0.5, D=1.0, set_point=1.0,Derivador=0, Integrador=0, max_Integrador=100, min_Integrator=-100):
		# Construtor - Inicia automaticamente quando um objeto da classe PID e instanciado
		self.Kp=P
		self.Ki=I
		self.Kd=D
		self.Derivador=Derivador
		self.Integrador=Integrador
		self.max_Integrador=max_Integrador
		self.min_Integrator=min_Integrator
		self.set_point=set_point
		self.error=0.0

	def update(self,current_value):
		"""
		Recebe um novo dado lido pelo sensor e retorna a resposta
		do controle PID para o sistema
		"""
		# Clculo do erro: Objetivo - Valor atual
		self.error = self.set_point - current_value
		# Clculo de K,P e D
		self.P_value = self.Kp * self.error
		self.D_value = self.Kd * ( self.error - self.Derivador)
		self.Derivador = self.error
		self.Integrador = self.Integrador + self.error
		# Checa se o valor do Integrador nao saturou
		if self.Integrador > self.max_Integrador:
			self.Integrador = self.max_Integrador
		elif self.Integrador < self.min_Integrator:
			self.Integrador = self.min_Integrator

		self.I_value = self.Integrador * self.Ki
		# Atualiza o valor da resposta 
		PID = self.P_value + self.I_value + self.D_value
		# Retorna o valor para a rotina que chamou o objeto
		return PID

	def setPoint(self,set_point):
		"""
		Atualiza o valor do set_point, caso um novo objetivo seja desejado
		Zera os valores do Integrador e do Derivador pois e como se o controle
		estivesse comeando novamente.
		"""
		self.set_point = set_point  # Atualiza o set_point (obejtivo)
		self.Integrador=0  # Zera o valor do Integrador
		self.Derivador=0   # Zera o valor do Derivador

	def getError(self): # Funao que retorna o erro atual
		return self.error

	def getIntegrator(self): # Altera o valor do integrador
		return self.Integrador # em tempo de execuao

	def getDerivator(self): # Altera o valor do derivalor
		return self.Derivador # em tempo de execuao

		
\end{lstlisting}
\doublespacing
%%%%%%%%%%%%%%%%%%%%%%%%%%%%%%%%%%%%%%%%%%%%%%%%%%%%%%%%%%%%%%%%%%%%%%%%%%%%%%%%%%%%%%%%%%%%%%%%%%%%%%%%%%%%%%%%%%%%%%

\section{Reconhecimento Visual}\label{alg:camera}

Programa em Python que tira uma foto com a câmera e aplica os filtros e reconhecimento de padrão para identificar se existem e quantos biscoitos estão na entrada do forno. Retorna a quantidade encontrada

Programa para Câmera Raspberry Pi:

\singlespacing
\lstset{language=Python}
\begin{lstlisting}
# -*- coding: latin-1 -*-
'''
Programa que tira fotos com a camera do raspberry pi e implementa um
filtro e detecta caracteristicas na imagem.
'''
import time
import os
import cv2
import numpy as np

print ' Inicio da captura ... '

biscoito = 0
arquivo  = 'foto'
extencao = '.jpg'
delay    = '-t 100 '
rotacao  = '-rot 180 '
tirafoto = 'raspistill '
final    = 100
i        = 1

while i <= final:   # Loop que se repete ate chegar biscoitos no inicio da esteira
    # comando pelo terminal para chamar a camera e salvar imagem
    os.system(tirafoto + delay + rotacao + '-o ' + arquivo + str(i) + extencao)
    time.sleep(2) # delay para garantir que a imagem foi salva
    if len(sys.argv)>1:
        imagem_nome = sys.argv[1]
    else:
        imagem_nome = arquivo + '.' + extencao
        # Chama a funcao de reconhecimento de imagem
        biscoito = trata_imagem(imagem_nome,6) 
    if biscoito > 0: # Caso tenham biscoitos na esteira
        print 'Iniciar Processo - Bolachas encontradas'
        # Chama a rotina de inicio do controle
        break # Sai do loop
    time.sleep(5)
\end{lstlisting}

\doublespacing

Programa Reconhecimento Imagem:

\singlespacing
\lstset{language=Python}
\begin{lstlisting}
# -*- coding: latin-1 -*-
import numpy as np
import cv2
import sys

def trata_imagem(imagem_pura,tipo):
    biscoitos = 0
    # borrar 11 x 11 pixels 
    imagem_blur       = cv2.blur(imagem_pura,(11,11)) 
    # Converte para tons de cinza
    imagem_gray       = cv2.cvtColor(imagem_blur, cv2.COLOR_BGR2GRAY) 
    # limiar adaptativo para imagem binaria
    imagem_thresh     = cv2.adaptiveThreshold(imagem_gray,255,1,0,29,0)
    # encontra o contorno das imagens binarias
    imagem_contorno   = imagem_thresh.copy()
    cv2.findContours(imagem_contorno,3,2)
    # Segundo metodo: Tranformando imagens do espectro RGB para HSV
    imagem_hsv        = cv2.cvtColor(imagem_blur,cv2.COLOR_BGR2HSV)
    # Filtrando a componente das cores (0 a 80)
    imagem_hsv_gray   = cv2.cvtColor(imagem_hsv, cv2.COLOR_BGR2GRAY)
    imagem_hsv_thresh = cv2.inRange(imagem_hsv,np.array((0, 80, 0)), np.array((80, 255, 255)))
    # invertendo as cores para o alimento ficar com valor 1 e fundo valor 0
    cv2.bitwise_not(imagem_hsv_thresh,imagem_hsv_thresh)
    imagem_hsv_blur   = cv2.blur(imagem_hsv_thresh,(3,3))
    # Tipo de imagem que sera salva (para acompanhar o tratamento)
    if   tipo == 0:
        depois = imagem_pura
    elif tipo == 1:
        depois = imagem_blur
    elif tipo == 2:
        depois = imagem_gray
    elif tipo == 3:
        depois = imagem_thresh
    elif tipo == 4:
        depois = imagem_contorno
    elif tipo == 5:
        depois = imagem_hsv_gray
    elif tipo == 6:
        depois = imagem_hsv_thresh
    elif tipo == 7:
        depois = imagem_hsv_blur 
    else:
        print 'value error'
    # Encontra circulos na imagem tratada
    circles = cv2.HoughCircles(depois,cv2.cv.CV_HOUGH_GRADIENT,2,80,param1=50,param2=35,minRadius=45,maxRadius=55)
    if circles is not None: # Se forem encontrados circulos na imagem:
        for i in circles[0,:]: # Loop que passa por todos os circulos encontrados
            # Desenha os circulos nos objetos encontrados
            cv2.circle(imagem_pura,(i[0],i[1]),i[2],(255,0,0),5)
            cv2.circle(depois,(i[0],i[1]),i[2],(255,0,0),5)
            biscoitos = biscoitos + 1
    # Descomentar linhas abaixo para mostrar resultado
    #cv2.imshow('resultado',depois)
    #cv2.waitKey(0)
    #cv2.destroyAllWindows()
    #cv2.imwrite('resultado.jpeg',depois)
    return biscoitos

if len(sys.argv)>1:
    imagem_nome = sys.argv[1]
    print '>1'
else:
    imagem_nome = 'antes.png'
    print '<=1'

imagem = cv2.imread(imagem_nome)   # Abre o arquivo com a imagem
if imagem == None:
    print 'erro ao abrir o arquivo'
else:
    biscoitos = trata_imagem(imagem,6) # chama a funcao de reconhecimento
    print 'imagem tratada,', biscoitos, 'biscoitos' 
\end{lstlisting}
\doublespacing
%%%%%%%%%%%%%%%%%%%%%%%%%%%%%%%%%%%%%%%%%%%%%%%%%%%%%%%%%%%%%%%%%%%%%%%%%%%%%%%%%%%%%%%%%%%%%%%%%%%%%%%%%%%%%%%%%%%%%%

\section{Interface}

Interface gráfica em python utilizando a biblioteca PyQt4. 

\singlespacing
\lstset{language=Python}
\begin{lstlisting}
from PyQt4 import QtCore, QtGui

try:
    _fromUtf8 = QtCore.QString.fromUtf8
except AttributeError:
    def _fromUtf8(s):
        return s

try:
    _encoding = QtGui.QApplication.UnicodeUTF8
    def _translate(context, text, disambig):
        return QtGui.QApplication.translate(context, text, disambig, _encoding)
except AttributeError:
    def _translate(context, text, disambig):
        return QtGui.QApplication.translate(context, text, disambig)

class Ui_MainWindow(object):
    def setupUi(self, MainWindow):
        MainWindow.setObjectName(_fromUtf8("MainWindow"))
        MainWindow.setEnabled(True)
        MainWindow.resize(800, 552)
        self.centralwidget = QtGui.QWidget(MainWindow)
        self.centralwidget.setObjectName(_fromUtf8("centralwidget"))
        self.qvtkWidget = QVTKWidget(self.centralwidget)
        self.qvtkWidget.setGeometry(QtCore.QRect(20, 210, 511, 151))
        self.qvtkWidget.setObjectName(_fromUtf8("qvtkWidget"))
        self.mplwidget = MatplotlibWidget(self.centralwidget)
        self.mplwidget.setEnabled(True)
        self.mplwidget.setGeometry(QtCore.QRect(20, 10, 511, 181))
        self.mplwidget.setObjectName(_fromUtf8("mplwidget"))
        self.lcdNumber = QtGui.QLCDNumber(self.centralwidget)
        self.lcdNumber.setGeometry(QtCore.QRect(70, 370, 101, 51))
        self.lcdNumber.setProperty("value", 25.0)
        self.lcdNumber.setObjectName(_fromUtf8("lcdNumber"))
        self.lcdNumber_2 = QtGui.QLCDNumber(self.centralwidget)
        self.lcdNumber_2.setGeometry(QtCore.QRect(250, 370, 101, 51))
        self.lcdNumber_2.setProperty("value", 25.0)
        self.lcdNumber_2.setObjectName(_fromUtf8("lcdNumber_2"))
        self.lcdNumber_3 = QtGui.QLCDNumber(self.centralwidget)
        self.lcdNumber_3.setGeometry(QtCore.QRect(430, 370, 101, 51))
        self.lcdNumber_3.setProperty("value", 25.0)
        self.lcdNumber_3.setObjectName(_fromUtf8("lcdNumber_3"))
        self.graphicsView = QtGui.QGraphicsView(self.centralwidget)
        self.graphicsView.setGeometry(QtCore.QRect(540, 10, 251, 181))
        self.graphicsView.setObjectName(_fromUtf8("graphicsView"))
        self.pushButton = QtGui.QPushButton(self.centralwidget)
        self.pushButton.setGeometry(QtCore.QRect(540, 290, 131, 41))
        self.pushButton.setObjectName(_fromUtf8("pushButton"))
        self.toolButton = QtGui.QToolButton(self.centralwidget)
        self.toolButton.setGeometry(QtCore.QRect(710, 230, 31, 21))
        self.toolButton.setObjectName(_fromUtf8("toolButton"))
        self.pushButton_2 = QtGui.QPushButton(self.centralwidget)
        self.pushButton_2.setGeometry(QtCore.QRect(540, 210, 131, 41))
        self.pushButton_2.setObjectName(_fromUtf8("pushButton_2"))
        self.label = QtGui.QLabel(self.centralwidget)
        self.label.setGeometry(QtCore.QRect(685, 210, 91, 20))
        self.label.setObjectName(_fromUtf8("label"))
        self.label_2 = QtGui.QLabel(self.centralwidget)
        self.label_2.setGeometry(QtCore.QRect(700, 290, 91, 20))
        self.label_2.setObjectName(_fromUtf8("label_2"))
        self.toolButton_2 = QtGui.QToolButton(self.centralwidget)
        self.toolButton_2.setGeometry(QtCore.QRect(710, 310, 31, 21))
        self.toolButton_2.setObjectName(_fromUtf8("toolButton_2"))
        self.label_3 = QtGui.QLabel(self.centralwidget)
        self.label_3.setGeometry(QtCore.QRect(630, 370, 91, 20))
        self.label_3.setObjectName(_fromUtf8("label_3"))
        self.label_4 = QtGui.QLabel(self.centralwidget)
        self.label_4.setGeometry(QtCore.QRect(550, 410, 31, 31))
        self.label_4.setObjectName(_fromUtf8("label_4"))
        self.label_5 = QtGui.QLabel(self.centralwidget)
        self.label_5.setGeometry(QtCore.QRect(740, 410, 31, 31))
        self.label_5.setObjectName(_fromUtf8("label_5"))
        self.label_6 = QtGui.QLabel(self.centralwidget)
        self.label_6.setGeometry(QtCore.QRect(660, 410, 31, 31))
        self.label_6.setObjectName(_fromUtf8("label_6"))
        self.dial = QtGui.QDial(self.centralwidget)
        self.dial.setGeometry(QtCore.QRect(70, 420, 101, 71))
        self.dial.setObjectName(_fromUtf8("dial"))
        self.dial_2 = QtGui.QDial(self.centralwidget)
        self.dial_2.setGeometry(QtCore.QRect(250, 420, 101, 71))
        self.dial_2.setObjectName(_fromUtf8("dial_2"))
        self.dial_3 = QtGui.QDial(self.centralwidget)
        self.dial_3.setGeometry(QtCore.QRect(430, 420, 101, 71))
        self.dial_3.setObjectName(_fromUtf8("dial_3"))
        self.progressBar = QtGui.QProgressBar(self.centralwidget)
        self.progressBar.setGeometry(QtCore.QRect(30, 370, 31, 121))
        self.progressBar.setProperty("value", 0)
        self.progressBar.setOrientation(QtCore.Qt.Vertical)
        self.progressBar.setObjectName(_fromUtf8("progressBar"))
        self.progressBar_2 = QtGui.QProgressBar(self.centralwidget)
        self.progressBar_2.setGeometry(QtCore.QRect(200, 370, 31, 121))
        self.progressBar_2.setProperty("value", 0)
        self.progressBar_2.setOrientation(QtCore.Qt.Vertical)
        self.progressBar_2.setObjectName(_fromUtf8("progressBar_2"))
        self.progressBar_3 = QtGui.QProgressBar(self.centralwidget)
        self.progressBar_3.setGeometry(QtCore.QRect(370, 370, 31, 121))
        self.progressBar_3.setProperty("value", 0)
        self.progressBar_3.setOrientation(QtCore.Qt.Vertical)
        self.progressBar_3.setObjectName(_fromUtf8("progressBar_3"))
        self.horizontalScrollBar = QtGui.QScrollBar(self.centralwidget)
        self.horizontalScrollBar.setGeometry(QtCore.QRect(550, 390, 221, 21))
        self.horizontalScrollBar.setMinimum(-5)
        self.horizontalScrollBar.setMaximum(5)
        self.horizontalScrollBar.setProperty("value", 0)
        self.horizontalScrollBar.setOrientation(QtCore.Qt.Horizontal)
        self.horizontalScrollBar.setObjectName(_fromUtf8("horizontalScrollBar"))
        self.lcdNumber_4 = QtGui.QLCDNumber(self.centralwidget)
        self.lcdNumber_4.setGeometry(QtCore.QRect(620, 440, 91, 41))
        self.lcdNumber_4.setObjectName(_fromUtf8("lcdNumber_4"))
        MainWindow.setCentralWidget(self.centralwidget)
        self.menubar = QtGui.QMenuBar(MainWindow)
        self.menubar.setGeometry(QtCore.QRect(0, 0, 800, 21))
        self.menubar.setObjectName(_fromUtf8("menubar"))
        MainWindow.setMenuBar(self.menubar)
        self.statusBar = QtGui.QStatusBar(MainWindow)
        self.statusBar.setObjectName(_fromUtf8("statusBar"))
        MainWindow.setStatusBar(self.statusBar)

        self.retranslateUi(MainWindow)
        QtCore.QObject.connect(self.dial, QtCore.SIGNAL(_fromUtf8("valueChanged(int)")), self.progressBar.setValue)
        QtCore.QObject.connect(self.dial_2, QtCore.SIGNAL(_fromUtf8("valueChanged(int)")), self.progressBar_2.setValue)
        QtCore.QObject.connect(self.dial_3, QtCore.SIGNAL(_fromUtf8("valueChanged(int)")), self.progressBar_3.setValue)
        QtCore.QObject.connect(self.horizontalScrollBar, QtCore.SIGNAL(_fromUtf8("valueChanged(int)")), self.lcdNumber_4.display)
        QtCore.QMetaObject.connectSlotsByName(MainWindow)

    def retranslateUi(self, MainWindow):
        MainWindow.setWindowTitle(_translate("MainWindow", "MainWindow", None))
        self.pushButton.setText(_translate("MainWindow", "Automatico", None))
        self.toolButton.setText(_translate("MainWindow", "...", None))
        self.pushButton_2.setText(_translate("MainWindow", "Liga", None))
        self.label.setText(_translate("MainWindow", "<html><head/><body><p>Imagem Padrão</p></body></html>", None))
        self.label_2.setText(_translate("MainWindow", "<html><head/><body><p>Perfil Padrão</p></body></html>", None))
        self.toolButton_2.setText(_translate("MainWindow", "...", None))
        self.label_3.setText(_translate("MainWindow", "<html><head/><body><p>Esteira - Velocidade</p></body></html>", None))
        self.label_4.setText(_translate("MainWindow", "<html><head/><body><p>Tras</p></body></html>", None))
        self.label_5.setText(_translate("MainWindow", "<html><head/><body><p>Frente</p></body></html>", None))
        self.label_6.setText(_translate("MainWindow", "<html><head/><body><p>Parado</p></body></html>", None))

from matplotlibwidget import MatplotlibWidget
from QVTKWidget import QVTKWidget

if __name__ == "__main__":
    import sys
    app = QtGui.QApplication(sys.argv)
    MainWindow = QtGui.QMainWindow()
    ui = Ui_MainWindow()
    ui.setupUi(MainWindow)
    MainWindow.show()
    sys.exit(app.exec_())
\end{lstlisting}
\doublespacing
