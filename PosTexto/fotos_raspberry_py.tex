\lstset{language=Python}
\begin{lstlisting}
# -*- coding: latin-1 -*-
'''
Programa que tira fotos com a camera do raspberry pi e implementa um
filtro e detecta caracteristicas na imagem.
'''
import time
import os
import cv2
import numpy as np

print ' Inicio da captura ... '

biscoito = 0
arquivo  = 'foto'
extencao = '.jpg'
delay    = '-t 100 '
rotacao  = '-rot 180 '
tirafoto = 'raspistill '
final    = 100
i        = 1

while i <= final:   # Loop que se repete ate chegar biscoitos no inicio da esteira
    # comando pelo terminal para chamar a camera e salvar imagem
    os.system(tirafoto + delay + rotacao + '-o ' + arquivo + str(i) + extencao)
    time.sleep(2) # delay para garantir que a imagem foi salva
    if len(sys.argv)>1:
        imagem_nome = sys.argv[1]
    else:
        imagem_nome = arquivo + '.' + extencao
        # Chama a funcao de reconhecimento de imagem
        biscoito = trata_imagem(imagem_nome,6) 
    if biscoito > 0: # Caso tenham biscoitos na esteira
        print 'Iniciar Processo - Bolachas encontradas'
        # Chama a rotina de inicio do controle
        break # Sai do loop
    time.sleep(5)
\end{lstlisting}
