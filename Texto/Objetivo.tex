\chapter{Objetivo}\label{objetivo}

Este trabalho tem como objetivo testar a seguinte hipótese:

\textit{"É possível o controle e automação de um forno experimental tipo túnel com informações de sensores e simulação computacional em tempo real utilizando computação embarcada de baixo custo"}

Para testar esta hipótese os seguintes objetivos específicos foram \textcolor{blue}{(azul)} ou serão \textcolor{red}{(vermelho)} desenvolvidos:

\begin{itemize}

    \item \textcolor{blue}{Modelagem matemática do comportamento térmico do forno.}
  
    \item \textcolor{blue}{Simulação do comportamento térmico utilizando diferenças finitas.}
  
    \begin{itemize}
    
        \item \textcolor{blue}{Teste da velocidade de convergência da simulação em função de várias técnicas de diferenças finitas.}
    
        \item \textcolor{blue}{Teste de tempo de execução em função da linguagem em que o algoritmo foi implementado.}
    
        \item \textcolor{red}{Levantar o perfil térmico do forno com sensores embarcados e comparar com o perfil simulado para validação.}
    
    \end{itemize}

    \item \textcolor{blue}{Implementação de um controle PID baseado na resposta da simulação e do conhecimento do perfil de temperatura desejado para o biscoito.}   
   
    \begin{itemize}
    
        \item \textcolor{blue}{Desenvolvimento de um algoritmo de controle PID em função de Kp, Ki e Kd.} 
    
        \item \textcolor{red}{Obter a resposta impulsiva da temperatura no forno e 	calcular a função de transferência (Temperatura em função da 	potência).}

        \item \textcolor{red}{Encontrar o Kp, Ki e Kd ótimos para o controle do forno.}
        
        \item \textcolor{red}{Comparar a função de transferência obtida com o perfil térmico em várias situações de operação do forno.}
    
    \end{itemize}

    \item \textcolor{blue}{Desenvolvimento de um software com interface gráfica amigável para controle do forno.}
    
    \item \textcolor{red}{Teste do sistema em vários dispositivos embarcados.}
 
\end{itemize}



