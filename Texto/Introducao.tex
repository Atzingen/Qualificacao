\chapter{Introdução}\label{introducao}
\pagestyle{fancyplain}
\pagenumbering{arabic}
No Brasil, a indústria de alimentos e bebidas é responsável por aproximadamente 9,5\% do Produto Interno Bruto (PIB), além de empregar 1,63 milhão de pessoas. Esta indústria manteve crescimento estável nos últimos 10 anos \citep{gov-ibge}, inclusive nos períodos de desaceleração econômica, fato este que, aliado a balança comercial positiva (em 2012 o Brasil exportou US\$ 43,4 bilhões e importou apenas US\$ 5,6 bilhões) faz deste setor um ponto estratégico de investimento tecnológico. Este investimento é necessário também em virtude da mudança no padrão de consumo do brasileiro nas últimas décadas, que está migrando de produtos in natura para processados – 56\% em 1980, 70\% em 1990 e 85\% em 2013 \citep{abia}.

O investimento em tecnologia para a produção de produtos processados, além de atender as novas demandas do mercado interno com produtos de saúde e bem-estar e comidas rápidas, pode agregar valor ao alimento exportado, já que o lucro com produtos processados é muito maior do que com insumos brutos. No entanto, apesar de possuir um grande parque industrial alimentício, o Brasil ainda possui carências no setor tecnológico que dá suporte a indústria. A maioria das empresas de grande porte que atuam no Brasil importam seus equipamentos mais sofisticados ou possuem seus setores de desenvolvimento tecnológico fora do País \citep{obstaculos}.

De forma geral, na indústria de alimentos o desenvolvimento tecnológico busca garantir um padrão de qualidade nos produtos, produzindo da forma mais eficiente e controlada possível. Para que isto possa ocorrer, os conceitos de controle, automação, modelagem e simulação devem estar integrados à indústria de alimentos, constituindo uma das principais áreas de pesquisa da Engenharia de Alimentos. De acordo com \citet{joao_da_silva}, como no Brasil esta área ainda é pouco desenvolvida, as grandes empresas muitas vezes buscam assistência tecnológica fora do País, diminuindo os lucros do setor e gerando no Brasil apenas a mão de obra barata, contribuindo para o crescimento de uma indústria de caráter regressivo.

O desenvolvimento tecnológico mundial da indústria alimentícia foi enorme nas últimas duas décadas, foram feitos progressos e descobertas em todas as etapas, começando pelo plantio, passando pelo processamento até o armazenamento \citep{challenges}. No entanto, no Brasil também é necessário que tal tecnologia esteja mais acessível às pequenas empresas, dando a elas o suporte para que possam ser mais eficientes e entregar um produto de maior qualidade de forma competitiva. O estudo de novos sensores, equipamentos e métodos na produção de alimentos é o desafio para o futuro, já que será necessário atender uma população cada vez maior, que também tem se tornado exigente quanto ao produto que consome.

Integrar sensores e atuadores para controlar de forma automática o processamento de um produto sempre foi um dos principais focos dos setores de Pesquisa e Desenvolvimento (P\&D) das indústrias de alimentos. De acordo com \citet{future-trends}, a importância da automação nos processos industriais aumentou dramaticamente nas últimas décadas, melhorando a qualidade e segurança dos produtos, reduzindo os custos de produção, gastos energéticos e emissão de poluentes. A utilização de modelos matemáticos e simulações integradas ao controle e automação cresceu notavelmente nos últimos 10 anos, principalmente devido ao aumento do poder computacional dos dispositivos eletrônicos, melhorando consideravelmente a automação industrial.

Padronizar a produção é o principal mecanismo para garantir a boa percepção dos clientes em relação aos produtos, especialmente na indústria de alimentos. A extinção da variação nos processos somente pode ser promovida com o estabelecimento de rotinas de produção que permitam a aquisição de informações e tratamento das mesmas, resultando em um controle que aumenta a eficiência da produção de forma que seja possível fornecer produtos com padrão de qualidade e evitar desperdícios na produção.

A etapa que apresenta mais perdas e problemas de controle é o aquecimento forçado, principalmente em fornos industriais. O monitoramento do ambiente interno de um forno de produção de alimentos é fundamental para padronizar algumas características importantes do alimento como a atividade de água, textura, coloração, crocânica e etc. O perfil de temperatura e tempo que um alimento fica exposto ao processo de cozimento são os principais fatores a alterarem tais características e portanto é essencial o desenvolvimento de ferramentas e dispositivos que auxiliem neste controle.

	A evolução dos dispositivos eletrônicos nos últimos 35 anos foi assombrosa, com o poder computacional aumentando cerca de 10.000 vezes entre 1978 e 2010 \citep{history-computer}. Além disso, o tamanho e preço dos dispositivos diminuíram sensivelmente, melhorando a facilidade de uso e conectividade. No entanto, os sistemas de automação industrial, principalmente no Brasil não tem acompanhado tal evolução e ainda são muito caros e de difícil uso, além de serem todos de tecnologia importada. Neste cenário, a utilização de sistemas embarcados de hardware aberto, como por exemplo o raspberry pi\texttrademark, que possui 1 Ghz de processamento e pode ser comprado por apenas \$35, pode ser uma alternativa para indústrias de pequeno e médio porte.
	
	Este trabalho tem como objetivo apresentar um software e hardware de controle e simulação em tempo real utilizando um sistema embarcado de hardware livre de baixo custo que garanta a qualidade na produção. O sistema foi projetado para um forno tipo túnel em escala reduzida para produção de biscoitos. 

